%-------------------- Packages --------------------%

\documentclass[11pt,a4paper]{article}
\usepackage[utf8]{inputenc}
\usepackage[french]{babel}
\usepackage[T1]{fontenc}
\usepackage{amsmath} %pour les formules maths
\usepackage{amsfonts}
\usepackage{amssymb}
\usepackage{graphicx} %pour inclure des images
\usepackage{hyperref} %pour les lien URLs
\usepackage{framed}
\usepackage[left=2cm,right=2cm,top=3cm,bottom=3cm]{geometry}
\usepackage{pifont} %pour les puces spéciales
\usepackage{listings} %pour écrire du code
\usepackage{xcolor} %pour la mise en couleur
\usepackage{multirow} %pour les tableaux
\usepackage{bclogo} %pour des boîtes texte avec un logo
\usepackage{eurosym}

%-------------------- New Colors --------------------%

%Basic

\definecolor{purple}{rgb}{0.6, 0.2, 0.9}

%Dark
\definecolor{dkgreen}{rgb}{0,.6,0}
\definecolor{dkblue}{rgb}{0,0,.6}
\definecolor{dkyellow}{cmyk}{0,0,.8,.3}
\definecolor{dkred}{rgb}{0.6, 0, 0}


%Light
\definecolor{ltblue}{rgb}{0.67, 0.85, 0.9}
\definecolor{ltred}{rgb}{0.8, 0.32, 0.32}
\definecolor{ltgrey}{rgb}{0.94,0.94,0.94}



%-------------------- Config lstlisting --------------------%

\lstset{
language=php,
basicstyle=\ttfamily\tiny, %
identifierstyle=\color{dkred}, %
keywordstyle=\color{dkblue}, %
stringstyle=\color{dkgreen}, %
commentstyle=\it\color{gray}, %
emph=[1]{php},
emphstyle=[1]\color{black},
emph=[2]{if,and,or,else},
emphstyle=[2]\color{dkyellow},
emph=[3]{imap_open, imap_close, imap_fetch_overview, imap_check, imap_list, imap_mail_move, imap_search, imap_fetchstructure			},
emphstyle=[3]\color{dkblue},
emph=[4]{string, int, double, float, private, public, static, bool, resource, array, object},
emphstyle=[4]\color{purple},
columns=flexible, %
tabsize=2, %
extendedchars=true, %
showspaces=false, %
showstringspaces=false, %
numbers=left, %
numberstyle=\tiny, %
breaklines=true, %
breakautoindent=true, %
captionpos=b, %
backgroundcolor=\color{ltgrey}
}


%-------------------- New Commands --------------------%

%Pour rédiger des QUESTIONS: la question est inscrite en noir et la réponse est automatiquement mise à la ligne, en vert, avec une puce spécifique
\newcommand{\quest}[2]{#1\newline\hspace*{.5cm}\textcolor{dkgreen}{\ding{228} #2}\paragraph{}}

%Pour rédiger un simple GLOSSAIRE: le terme est séparé de sa définition par ':'
\newcommand{\gloss}[2]{\paragraph{}$\square$\hspace*{.2cm} \textbf{\texttt{{#1}}}: \hspace*{.3cm}#2}

%Pour rédiger une définition de DICTIONNAIRE
\newcommand{\dict}[3]{\paragraph{}$\square$\hspace*{.2cm} \textbf{\texttt{{#1}}}: \hspace*{.3cm}#3 \newline\textit{Dom: #2}}

%Pour rédiger une INFORMATION -----> CHERCHER PACKAGE FANCYBOX 
%Rem: \colorbox est chargée via le package "graphicx"
\newcommand{\info}[1]{\paragraph{}\colorbox{ltblue}{
	\begin{minipage}{.9\linewidth}#1\end{minipage}} }
%PACKAGE BCLOGO pour obtenir des boîtes dans lesquelles mettre un logo en plus du texte
%une alternative de minipage est "parbox"
%Site: "latex-howto.be"
%le package "framed" permet d'étendre les box sur plusieurs pages
%il existe la commande \newsavebox{\name} qui permet d'enregistrer le pattern d'une box et de la réutiliser à tout moment 
%	via \usebox{\name}. \savebow permet de définir le pattern
%Se  renseigner sur la création de nouveaux logos (fichiers metapost, eps


%Pour rédiger une REMARQUE


%pour régider un ATTENTION



%Pour rédiger une NOTE
\newcounter{NoteCounter}\setcounter{NoteCounter}{1} %Compteur qui s'incrémente à chaque nouvelle note

\newcommand{\note}[3]{\textcolor{dkblue}{\paragraph{\ding{229} Note \theNoteCounter\addtocounter{NoteCounter}{1}} 
\begin{itemize}\item[$\cdot$]\textsc{Date: }#1\item[$\cdot$]\textsc{Sujet: }\textbf{#2}\end{itemize} \paragraph{}#3}}

\newcommand{\snote}[4]{\textcolor{dkred}{\paragraph{\ding{229} Note \theNoteCounter\addtocounter{NoteCounter}{1}} \begin{itemize}\item[$\cdot$]\textsc{Date: }#1\item[$\cdot$]\textsc{Intervenant: }#2\item[$\cdot$]\textsc{Sujet: }\textbf{#3}\end{itemize} \paragraph{}#4}}
