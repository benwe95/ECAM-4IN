\documentclass{sig-alternate-05-2015}
%-------------------- Packages --------------------%

\documentclass[11pt,a4paper]{article}
\usepackage[utf8]{inputenc}
\usepackage[french]{babel}
\usepackage[T1]{fontenc}
\usepackage{amsmath} %pour les formules maths
\usepackage{amsfonts}
\usepackage{amssymb}
\usepackage{graphicx} %pour inclure des images
\usepackage{hyperref} %pour les lien URLs
\usepackage{framed}
\usepackage[left=2cm,right=2cm,top=3cm,bottom=3cm]{geometry}
\usepackage{pifont} %pour les puces spéciales
\usepackage{listings} %pour écrire du code
\usepackage{xcolor} %pour la mise en couleur
\usepackage{multirow} %pour les tableaux
\usepackage{bclogo} %pour des boîtes texte avec un logo
\usepackage{eurosym}

%-------------------- New Colors --------------------%

%Basic

\definecolor{purple}{rgb}{0.6, 0.2, 0.9}

%Dark
\definecolor{dkgreen}{rgb}{0,.6,0}
\definecolor{dkblue}{rgb}{0,0,.6}
\definecolor{dkyellow}{cmyk}{0,0,.8,.3}
\definecolor{dkred}{rgb}{0.6, 0, 0}


%Light
\definecolor{ltblue}{rgb}{0.67, 0.85, 0.9}
\definecolor{ltred}{rgb}{0.8, 0.32, 0.32}
\definecolor{ltgrey}{rgb}{0.94,0.94,0.94}



%-------------------- Config lstlisting --------------------%

\lstset{
language=php,
basicstyle=\ttfamily\tiny, %
identifierstyle=\color{dkred}, %
keywordstyle=\color{dkblue}, %
stringstyle=\color{dkgreen}, %
commentstyle=\it\color{gray}, %
emph=[1]{php},
emphstyle=[1]\color{black},
emph=[2]{if,and,or,else},
emphstyle=[2]\color{dkyellow},
emph=[3]{imap_open, imap_close, imap_fetch_overview, imap_check, imap_list, imap_mail_move, imap_search, imap_fetchstructure			},
emphstyle=[3]\color{dkblue},
emph=[4]{string, int, double, float, private, public, static, bool, resource, array, object},
emphstyle=[4]\color{purple},
columns=flexible, %
tabsize=2, %
extendedchars=true, %
showspaces=false, %
showstringspaces=false, %
numbers=left, %
numberstyle=\tiny, %
breaklines=true, %
breakautoindent=true, %
captionpos=b, %
backgroundcolor=\color{ltgrey}
}


%-------------------- New Commands --------------------%

%Pour rédiger des QUESTIONS: la question est inscrite en noir et la réponse est automatiquement mise à la ligne, en vert, avec une puce spécifique
\newcommand{\quest}[2]{#1\newline\hspace*{.5cm}\textcolor{dkgreen}{\ding{228} #2}\paragraph{}}

%Pour rédiger un simple GLOSSAIRE: le terme est séparé de sa définition par ':'
\newcommand{\gloss}[2]{\paragraph{}$\square$\hspace*{.2cm} \textbf{\texttt{{#1}}}: \hspace*{.3cm}#2}

%Pour rédiger une définition de DICTIONNAIRE
\newcommand{\dict}[3]{\paragraph{}$\square$\hspace*{.2cm} \textbf{\texttt{{#1}}}: \hspace*{.3cm}#3 \newline\textit{Dom: #2}}

%Pour rédiger une INFORMATION -----> CHERCHER PACKAGE FANCYBOX 
%Rem: \colorbox est chargée via le package "graphicx"
\newcommand{\info}[1]{\paragraph{}\colorbox{ltblue}{
	\begin{minipage}{.9\linewidth}#1\end{minipage}} }
%PACKAGE BCLOGO pour obtenir des boîtes dans lesquelles mettre un logo en plus du texte
%une alternative de minipage est "parbox"
%Site: "latex-howto.be"
%le package "framed" permet d'étendre les box sur plusieurs pages
%il existe la commande \newsavebox{\name} qui permet d'enregistrer le pattern d'une box et de la réutiliser à tout moment 
%	via \usebox{\name}. \savebow permet de définir le pattern
%Se  renseigner sur la création de nouveaux logos (fichiers metapost, eps


%Pour rédiger une REMARQUE


%pour régider un ATTENTION



%Pour rédiger une NOTE
\newcounter{NoteCounter}\setcounter{NoteCounter}{1} %Compteur qui s'incrémente à chaque nouvelle note

\newcommand{\note}[3]{\textcolor{dkblue}{\paragraph{\ding{229} Note \theNoteCounter\addtocounter{NoteCounter}{1}} 
\begin{itemize}\item[$\cdot$]\textsc{Date: }#1\item[$\cdot$]\textsc{Sujet: }\textbf{#2}\end{itemize} \paragraph{}#3}}

\newcommand{\snote}[4]{\textcolor{dkred}{\paragraph{\ding{229} Note \theNoteCounter\addtocounter{NoteCounter}{1}} \begin{itemize}\item[$\cdot$]\textsc{Date: }#1\item[$\cdot$]\textsc{Intervenant: }#2\item[$\cdot$]\textsc{Sujet: }\textbf{#3}\end{itemize} \paragraph{}#4}}


\makeatletter
\def\@copyrightspace{\relax}
\makeatother

\begin{document}
%General Information
\setcopyright{acmcopyright}
\title{La biométrie par reconnaissance faciale}
%\subtitle{Analyse et découverte d'une technologie}
\numberofauthors{1}
\author{
Wéry Benoît\\
       \affaddr{ECAM - 1e Master Informatique}\\
       \affaddr{1200 Bruxelles, Belgique}\\
       \affaddr{\today}\\
}
\maketitle

\begin{abstract}
%Prévoir 1 page (2 colonnes) pour Abstract + Introduction + Conclusion -> reste seulement 3 pages (6 colonnes)
\end{abstract}

\section{Introduction}
\textit{Bref rappel historique, motivations, utilité, ... Présentation du découpage de l'article}
\section{La biométrie}
%\textit{Une donnée est dite biométrique si elle permet l'identification d'une personne sur base de ce qu'elle est, qu'il s'agisse d'une caractéristique physiologique ou comportementale. Ainsi, on peut parler du visage comme étant une donnée biométrique.}
La biométrie, qui signifie "mesure du vivant", désigne dans notre contexte \textit{"l'ensemble des procédés de reconnaissance d'une personne par certaines de ses caractéristiques physiques ou comportementales"}.\cite{Xmisc_3}. Il s'agit donc d'utiliser des informations, telles que: l'empreinte digitale, l'iris, le visage, la démarche, ... afin de pouvoir identifier ou confirmer l'identité d'un sujet humain.
%\begin{itemize}
%\item[$\cdot$]\textsc{universelles} - peuvent être analyser sur chaque individu
%\item[$\cdot$]\textsc{uniques} - sont différentes pour chaque individu
%\item[$\cdot$]\textsc{invariables} - ne changent que très peu, voir pas du tout, au cours du temps
%\item[$\cdot$]\textsc{enregistrables} - peuvent être stockées de façon numérique
%\item[$\cdot$]\textsc{mesurables} - peuvent être comparées
%\end{itemize}

\subsection{Systèmes biométriques}
%\textit{Les composants principaux et les différents modes (enrôlement, identificaiton, authentification)}
Un système biométrique fonctionne sur la comparaison de deux fichiers, issus de données biométriques, afin de déterminer leur taux de similitude.
\\
Dans un tel système, une première phase dite d'\textit{enrôlement} permet de récupérer la donnée et de l'enregistrer de façon numérique en BDD sous forme d'un modèle mathématique, que l'on appelle \textit{"signature"} ou \textit{"gabarit"}. On distingue ensuite deux modes de comparaison des modèles (voir \ref{id-vs-auth}): 
\begin{itemize}
\item[$\cdot$]l'\textsc{authentification} - comparaison 1:1
\item[$\cdot$]l'\textsc{identification} - comparaison 1:N
\end{itemize}
\begin{figure}[h!]
\includegraphics[scale=.45]{images/systeme-biometrique.png}
\caption{Modules d'un système biométrique}
\end{figure}

\subsection{Aspects sécurité}
Les systèmes biométriques sont particulièrement appréciés pour augmenter la sécurité des processus de vérification.
\\En effet, un code PIN ou un mot de passe peuvent être facilement trouvés selon leur niveau de complexité. Les données biométriques, quant à elles, présentent les avantages suivants, elles sont: universelles, uniques, invariables, enregistrables et mesurables.
\\Ainsi, leur utilisation complique le piratage ainsi que l'usurpation d'identité. De plus, un système biométrique ne nécessite plus de retenir un code. Pour ces raisons, ils sont de plus en plus utilisés dans les processus qui nécessitent la vérification de l'utilisateur.

\subsection{Critères de performances et comparaison des technologies}
Les éléments essentiels qui déterminent la qualité d'un tel système sont: la \textit{donnée}, le \textit{capteur} - nécessité d'obtenir un modèle analysable de bonne résolution - et les \textit{algorithmes} (détection, analyse, comparaison).
%\textit{Intrusivité, fiabilité, coût, effort -> la place du visage d'en tout ça...}
\paragraph{}
Toutes les caractéristiques biométriques exploitables ne se valent pas mais elles peuvent être comparées selon différents critères tels que: l'\textit{intrusivité}, la \textit{fiabilité}, le \textit{coût} ou encore l'\textit{effort} (contribution du sujet lors de son analyse).
\paragraph{}
Ainsi, par exemple, si les empreintes digitales et l'iris sont meilleures que la reconnaissance du visage en termes de performances, cette dernière technique, quant à elle, est jugée moins intrusive et moins contraignante pour l'utilisateur. En effet, elle ne nécessite pas la coopération du sujet car il peut être identifié à distance. De plus les capteurs utilisés peuvent être relativement bons marché, puisqu'il s'agit dans le plus simple des cas d'un appareil photo ou d'une caméra.
\\Néanmoins, comme nous allons le voir, la reconnaissance faciale est sujette à diverses contraintes qui compliquent l'obtention d'une information de qualité et son analyse.

\subsection{Evaluation de la fiabilité}
Comme cela a été dit, un système biométrique évalue le taux de similitude entre deux modèles pour authentifer un individu. Or, il est impossible d'obtenir une coïncidence de 100\% entre deux signatures. Dès lors, il faut fixer des seuils d'acception, qui permettent de quantifier les performances d'un système selon les facteurs suivants \cite{Xmisc_2}:
\begin{itemize}
\item[$\cdot$]\textsc{TFR} - Taux de Faux Rejets : pourcentage d'individus rejetés alors qu'ils devraient être acceptés
\item[$\cdot$]\textsc{TFA} - Taux de Fausses Acceptation : pourcentage d'individus acceptés alors qu'ils devraient être rejetés
\item[$\cdot$]\textsc{TEE} - Taux d'Egale Erreur : point d'équivalence des erreurs. Il s'agit de l'intersection des deux autres courbes, qui est utilisée pour mesurer la performance de l'algorithme.
\end{itemize}
 (voir annexe \ref{perfo-systeme} pour les courbes)
\section{La reconnaissance faciale}
\textit{Présentation de qqes résultats fondamentaux sur des recherches en cognition et reconnaissance faciale du visage}
\subsection{Détection de visage}
\textit{Les difficultés rencontrées: Variations de la pose, changement d'éclairage (luminosité), expressions faciales, occultations,...\paragraph{}Les différentes méthodes: les "connaissances acquises", le "template matching", "l'apparence", les "caractéristiques invariantes"\paragraph{}Nécessité d'avoir une image de qualité -> dépend des capteurs}
\subsection{Prétraitement ou normalisation}
\textit{Méthodes globales (ACP et ADL) vs méthodes locales (localisation de points caractéristiques et partition du visage en régions caractéristiques). Eigenface, stéréovision}\subsection{Reconnaissance}
\textit{Exploitation des caractéristiques extraites, création d'une signature numérique et mise en correspondance avec les modèles de la DB ou le modèle vérifié\paragraph{}
Difficultés: causes inter-sujets (ressemblance entre modèles) et intra-sujet (ci-dessus)}

\subsection{Techniques de reconnaissance 2D et 3D}
\subsection{Mesure de la performance}
\textit{Mise à disposition de bases de données}


\section{Exemples d'utilisation}
\textit{Les applications de la reconnaissance faciale sont nombreuses: depuis la sécurité des systèmes, jusqu'à la modélisation d'animation 3D en passant par la recherche de suspects.}
\textit{Nous retiendrons ici deux exemples, chacun étant lié à un mode d'utilisation de la biométrie tel que cité précédemment: l'identification et l'authentification}.
\subsection{Les portails de sécurité dans les aéroports}
\textit{Ex: aéroport de Francfort - Contrôle des passagers automatisé par la reconnaissance de visages.}
\subsection{La FaceID de Apple}
\subsection{Mais encore... quel futur pour la reconnaissance faciale}
\textcolor{dkblue}{\textit{Compte tenu des progrès fait en matière de reconnaissance faciale et de son intégration dans des systèmes tels que les smartphones, quelles applications pourrait-on envisager à l'avenir grâce à une telle technologie?}}
\paragraph{}
\textit{Affichages publicitaires intelligents dont le contenu est adapté par la reconnaissance du visage.}
%\section{Les aspects liés à la sécurité}

\subsection{Impact par rapport aux anciens systèmes}
\subsection{Les risques potentiels}
\subsection{Législation}
\subsection{Questions éthiques... faut-il avoir "peur" de la reconnaissance faciale}
\textcolor{dkblue}{\textit{Les technologies biométriques stockent des informations très personnelles sur les individus comme ses empreintes digitales, ses données faciales, ... dès lors, cela soulève certaines questions quant à la protection de notre vie privée et la sécurité de ces informations. Il y a-t-il des risques potentiels à fournir tant de données à des "inconnus" , si oui lesquels? Dans quelles mesures peut-on accepter que des systèmes (sites, ) collectent autant de données sensibles à notre égard? Quelle serait-la prochaine étape?}}

\section{Conclusion}
Au fil de ces dernières décennies, les techniques de reconnaissance de visages se sont fortement améliorées proposant nombre de méthodes analysant des modèles 2D et 3D, avec des performances différentes et des dispositifs physiques plus ou moins onéreux. De plus, les nouvelles technologies ont permis de miniaturiser les capteurs de telle sorte que maintenant ils puissent même être embarqués dans des appareils mobiles.
\\Ces prouesses permettent ainsi d'obtenir des systèmes biométriques d'authentification de plus en plus performants qui améliorent la sécurité de nos appareils ou des mécanismes d'accès.
\\La reconnaissance de visages va également permettre à l'avenir de sécuriser des lieux publics, où circulent de nombreuses personnes, en traquant des suspects.
\\Toutefois, il est à noter que si les technologies biométriques représentent des solutions pratiques incontournables à l'heure du numérique et du "tout connecté", certains systèmes stockent des informations très personnelles sur les individus comme leurs empreintes digitales, leurs données faciales, leur comportement... Dès lors, cela peut soulever certaines questions quant à la protection de notre vie privée et la sécurité de ces informations. Il y a-t-il des risques potentiels à nous dévoiler autant à des "inconnus"? Serons-nous un jour contraints par les autorités de fournir l'ensemble de nos caratéristiques biométriques pour constituer des bases de données nationales? Dans quelle mesure peut-on accepter que des systèmes nous surveillent en public (\textit{"pour notre bien et la sécurité"})? Autant de questions pour lesquelles chacun à la liberté de se faire sa propre opinion mais probablement qu'il faudra un jour tout simplement accepter d'évoluer avec son temps et de profiter avant tout des bénéfices qu'offre ces nouvelles technologies, en échange de l'intrusion quelles occasionnent.

\bibliographystyle{unsrt}
\bibliography{biblio}

\appendix
\section{Evaluation du taux de performance des systèmes biométriques}
\begin{figure}[h!]
\center\includegraphics[scale=.5]{images/performances-systeme}\label{perfo-systeme}
\caption{URL [...]}
\end{figure}
\section{Comparaison des méthodes locales et globales de reconnaissance 2D}
\begin{figure}[h!]
\center\includegraphics[scale=.4]{images/locales-vs-globales}\label{locales-vs-globales}
\caption{Tableau comparatif issu de \cite{Xphdthesis_1}}
\end{figure}

\end{document}
