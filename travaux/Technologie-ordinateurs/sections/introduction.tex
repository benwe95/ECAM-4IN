\section{Introduction}
%\textit{Bref rappel historique, motivations, utilité, ... Présentation du découpage de l'article}
Si la reconnaissance de visages est une tâche complètement anodine pour le cerveau humain, il n'en va pas de même pour les systèmes informatiques. Bien au contraire, ce processus s'avère être un réel défi technologique mélangeant techniques de capture d'images et algorithmes de traitements. \\
Pourtant, la reconnaissance faciale automatique est très appréciée en tant que technique biométrique et utilisée dans de nombreuses applications. C'est pourquoi, depuis la fin des années 1970, les travaux de recherches ont permis de mettre au point de nouvelles méthodes qui se sont succédées et améliorées avec le temps pour passer de simples processus de traitement 2D à des systèmes capables d'analyser un visage en 3D et en temps réel, le tout dans des systèmes de plus en plus miniaturisés.
\paragraph{}
Alors, quelle est la place de la reconnaissance faciale par rapport aux différentes techniques biométriques existantes? Quelles sont les méthodes les plus connues et quelles applications les utilisent?\\ C'est ce que nous tenterons de découvrir à travers cet article.
\paragraph{}
Dans un premier, la notion de biométrie sera expliquée ainsi que le fonctionnement général des systèmes biométriques. On verra également pourquoi il est intéressant d'utiliser des caractéristiques de l'être humain pour améliorer les aspects de sécurité d'un système.
\\
Ensuite, la reconnaissance faciale sera présentée en suivant la logique de ses trois étapes caractéristiques, à savoir: la \textit{détection}, la \textit{normalisation} et la \textit{reconnaissance 2D et 3D}.
\\
Enfin, deux exemples seront présentés pour mettre en avant des cas d'utilisation différents des systèmes biométriques par reconnaissance de visage: la détection et l'authentification.