\section{Conclusion}
Au fil de ces dernières décennies, les techniques de reconnaissance de visages se sont fortement améliorées proposant nombre de méthodes analysant des modèles 2D et 3D, avec des performances différentes et des dispositifs physiques plus ou moins onéreux. De plus, les nouvelles technologies ont permis de miniaturiser les capteurs de telle sorte que maintenant ils puissent même être embarqués dans des appareils mobiles.
\\Ces prouesses permettent ainsi d'obtenir des systèmes biométriques d'authentification de plus en plus performants qui améliorent la sécurité de nos appareils ou des mécanismes d'accès.
\\La reconnaissance de visages va également permettre à l'avenir de sécuriser des lieux publics, où circulent de nombreuses personnes, en traquant des suspects.
\\Toutefois, il est à noter que si les technologies biométriques représentent des solutions pratiques incontournables à l'heure du numérique et du "tout connecté", certains systèmes stockent des informations très personnelles sur les individus comme leurs empreintes digitales, leurs données faciales, leur comportement... Dès lors, cela peut soulever certaines questions quant à la protection de notre vie privée et la sécurité de ces informations. Il y a-t-il des risques potentiels à nous dévoiler autant à des "inconnus"? Serons-nous un jour contraints par les autorités de fournir l'ensemble de nos caratéristiques biométriques pour constituer des bases de données nationales? Dans quelle mesure peut-on accepter que des systèmes nous surveillent en public (\textit{"pour notre bien et la sécurité"})? Autant de questions pour lesquelles chacun à la liberté de se faire sa propre opinion mais probablement qu'il faudra un jour tout simplement accepter d'évoluer avec son temps et de profiter avant tout des bénéfices qu'offre ces nouvelles technologies, en échange de l'intrusion quelles occasionnent.