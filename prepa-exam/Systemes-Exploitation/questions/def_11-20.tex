%--- 11 -------------------------------%
\item\quest{Définissez ce qu’est une section critique et dans quelle situation l’existence d’une telle section peut poser problème (illustrez avec un exemple). Citez et expliquez une solution que l’on peut
apporter pour protéger de telles sections en synchronisant des processsus.}
{}


%--- 12 -------------------------------%
\item\quest{Définissez ce qu’est un deadlock et dans quelles situations il peut se produire (illustrez avec un exemple). Comment peut-on détecter, prévenir et se remettre d’un deadlock ?}
{}


%--- 13 -------------------------------%
\item\quest{Citez et expliquez les différents moyens que l’on peut mettre en oeuvre pour faire communiquer
entre eux des processus coopératifs ? Quels sont les avantages et inconvénients de ces deux
moyens de communication ?}
{}


%--- 14 -------------------------------%
\item\quest{Expliquez ce qu’est le principe d’adressage ? Comment l’unité mémoire intervient-elle ? Définissez et expliquez le principe de liaison des adresses.}
{}


%--- 15 -------------------------------%
\item\quest{En quoi consiste le swapping ? Citez et expliquez les différentes formes de swapping qu’il existe, et comparez-les.}
{}


%--- 16 -------------------------------%
\item\quest{Citez, expliquez et comparez les différentes stratégies d’allocation de la mémoire que l’on peut mettre en place. Quels sont les risques de fragmentation liés à ces stratégies ?}
{}


%--- 17 -------------------------------%
\item\quest{Expliquez le principe de la segmentation. En quoi est-il plus proche de la pensée du programmeur? Comment les adresses sont-elles construites lorsqu’on utilise la segmentation ?}
{}


%--- 18 -------------------------------%
\item\quest{Expliquez le principe de la pagination. Quels sont ses avantages par rapport à la segmentation? Comment les adresses sont-elles construites lorsqu’on utilise la pagination ? Comment peut-on améliorer les performances de la pagination à l’aide d’une mémoire spécialisée ?}
{}


%--- 19 -------------------------------%
\item\quest{Citez et expliquez les différentes structures que l’on peut utiliser pour stocker la table des pages.
Quels sont les avantages et inconvénients de ces structures ? Dans quelles situations sont-elles
plus adaptées ?}
{}


%--- 20 -------------------------------%
\item\quest{Définissez et expliquez les différences et les liens entre mémoire physique et mémoire virtuelle. Comment le système d’exploitation fait-il le lien entre ces deux mémoires, pour les différents processus ?}
{}