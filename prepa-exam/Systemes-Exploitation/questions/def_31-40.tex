%--- 31 -------------------------------%
\item\quest{Citez et expliquez la notion de machine virtuelle et quelles sont les deux types de machines
virtuelles qu’il est possible de mettre en oeuvre.}
{}


%--- 32 -------------------------------%
\item\quest{Quelles sont les différences entre moniteur de machines virtuelles et système d’exploitation ?
En particulier, quelles sont les opérations qu’ils doivent tous les deux faire et celles qui les
distinguent.}
{}


%--- 33 -------------------------------%
\item\quest{Quelles sont les similitudes et différences entre les deux types d’hyperviseur. Quels sont les
avantages et inconvénients de ces deux types d’hyperviseur ?}
{}


%--- 34 -------------------------------%
\item\quest{Quelles sont les différences entre le kernel Linux, un système Linux et une distribution Linux. Mettez votre comparaison en parallèle avec les trois composants principaux de Linux qui sont en
accord avec Unix.}
{}


%--- 35 -------------------------------%
\item\quest{Définissez et expliquez ce que sont les modules kernel. En particulier, expliquez à quoi ils servent, en quoi ils permettent d’obtenir un kernel minimal et en quoi ils facilitent le développement de
drivers.}
{}


%--- 36 -------------------------------%
\item\quest{La création d’un nouveau processus en Linux passe par deux appels systèmes. Citez-les et
expliquez leur rôle. Quels avantages apporte un tel choix de design en comparaison à Windows où
un nouveau processus est directement créé avec un unique appel à l’appel système CreateProcess.}
{}

%--- 37 -------------------------------%
\item\quest{Expliquez le modèle CIA+ en détaillant chacun des objectifs visés.}
{}


%--- 38 -------------------------------%
\item\quest{Donnez et expliquez trois principes de conception pour le développement de mécanismes de
protection.}
{}


%--- 39 -------------------------------%
\item\quest{Expliquez comment construire un certificat de clé publique, et comment on peut vérifier l’authenticité
de la clé.}
{}

%--- 40 -------------------------------%
\item\quest{Expliquez ce qu’apporte l’utilisation du « sel »dans le stockage de mots de passe.}
{}