%--- 1 -------------------------------%
\item\quest{Citez, expliquez et comparez différentes approches possibles pour structurer le code d’un système d’exploitation. Quels sont les avantages et inconvénients de ces différentes approches en fonction
du système informatique à opérer ? Argumentez.}
{}

%--- 2 -------------------------------%
\item\quest{Définissez ce que sont les threads et comparez-les par rapport aux processus. Que permettent-ils de plus, par rapport à un système d’exploitation ne proposant que des processus ? Confondre
les threads et les processus en un seul type d’entité comme le fait Linux est-il intéressant ?
Argumentez.}
{}


%--- 3 -------------------------------%
\item\quest{Citez, expliquez et comparez différentes stratégies d’ordonnancement que l’on peut mettre en oeuvre (FCFS, SJF, priorité, RR, files multi-niveaux). Comment peut-on les comparer ? Y a-t-il
un meilleur algorithme dans l’absolu ? Argumentez.}
{}


%--- 4 -------------------------------%
\item\quest{L’acquisition de ressources par des processus peut conduire le système dans des états de deadlock. Il existe plusieurs stratégies permettant d’éviter ou de se remettre d’un deadlock. Quelles
sont-elles et en quoi elles sont plus ou moins adaptées selon le type de système informatique ?
Argumentez.}
{}


%--- 5 -------------------------------%
\item\quest{L’utilisation de threads en lieu et place de processus facilite-t-il la communication entre ces entités ? Est-on limités aux mêmes mécanismes de communication ? Où y gagne-t-on et que
perd-on ? Argumentez.}
{}


%--- 6 -------------------------------%
\item\quest{Citez, expliquez et comparez différentes stratégies de remplacement des pages que l’on peut
mettre en oeuvre (FIFO, optimal, LRU, approximation de LRU, LFU, MFU). Comment peut-on
les comparer ? Y a-t-il un meilleur algorithme dans l’absolu ? Argumentez.}
{}


%--- 7 -------------------------------%
\item\quest{Un système d’exploitation peut adopter plusieurs attitudes différentes suite à l’apparition
d’un deadlock. Citez-les et expliquez en quoi elles sont plus adaptées en fonction du système
d’exploitation à opérer. Argumentez.}
{}


%--- 8 -------------------------------%
\item\quest{Citez, expliquez et comparez différentes stratégies d’ordonnancement de disque l’on peut mettre en oeuvre (FCFS, SSTF, SCAN, C-SCAN, LOOK, C-LOOK). Comment peut-on les comparer ?
Y a-t-il un meilleur algorithme dans l’absolu ? Argumentez.}
{}


%--- 9 -------------------------------%
\item\quest{Une virtualisation est réussie si le système d’exploitation est incapable de savoir s’il tourne sur une machine physique ou sur une virtuelle. Expliquez comment un système d’exploitation
pourrait se rendre compte qu’il est trompé et ce qu’on peut mettre en oeuvre pour le tromper.}
{}


%--- 10 -------------------------------%
\item\quest{Dans plusieurs situations (ordonnancement de processus, remplacement de pages, ordonnancement
d’opérations disque), plusieurs algorithmes sont possibles. Y a-t-il à chaque fois un algorithme
qui est le meilleur dans tous les cas ? Comment peut-on comparer ces algorithmes ? Sont-ils
bon ou mauvais selon le système informatique à opérer ? Comment choisir du mieux possible
l’algorithme à utiliser étant donné un système d’exploitation ? Argumentez.}
{}
