%--- 1 -------------------------------%
\item\quest{Définissez ce qu’est le système d’exploitation d’un système informatique. En particulier, identifiez ses buts et les abstractions du système informatique qu’il permet d’opérer. Présentez également
la structure d’un système informatique et où il s’y situe.}
{}


%--- 2 -------------------------------%
\item\quest{Quelles sont les principales étapes qui sont franchies lors du démarrage d’un système informatique ? Citez-les et identifiez la fonction de chacune des étapes, c’est-à-dire le rôle des actions réalisées pour le fonctionnement du système informatique.}
{}


%--- 3 -------------------------------%
\item\quest{Citez et expliquez des exemples de services offerts par un système d’exploitation. Comment
peut-on les classer selon le type destinataire du service ? Citez et expliquez également les trois
aspects sur lesquels il joue le rôle de coordinateur.}
{}


%--- 4 -------------------------------%
\item\quest{Qu’est-ce-que la multiprogrammation ? Quelles conséquences cette tendance a-t-elle eue sur le
développement des systèmes d’exploitation ?}
{}


%--- 5 -------------------------------%
\item\quest{Citez et expliquez les deux modes d’exécution possible d’un programme. Qu’est-ce-qu’un appel
système et comment se déroule l’exécution d’un tel appel ?}
{}


%--- 6 -------------------------------%
\item\quest{Définissez le concept de processus et détaillez les différentes opérations que l’on peut réaliser depuis sa création jusque sa terminaison. Citez et expliquez quels sont les différents états possibles d’un processus.}
{}


%--- 7 -------------------------------%
\item\quest{Comment et où le système d’exploitation maintient-il de l’information par rapport aux processus? Quel genre d’informations garde-t-il et pourquoi ?}
{}


%--- 8 -------------------------------%
\item\quest{Quelles sont les différences entre parallélisme et concurrence ? Comment les obtient-on dans un système informatique donné ? Le hardware disponible est-il un facteur déterminant dans le choix
entre parallélisme et concurrence lors du développement d’un programme ?}
{}


%--- 9 -------------------------------%
\item\quest{Citez et caractérisez les deux types de threads en fonction du niveau où le support des threads est fourni. Citez, expliquez et comparez les différents mappings que l’on peut réaliser entre ces
deux types de thread.}
{}


%--- 10 -------------------------------%
\item\quest{Définissez ce qu’est et ce que fait l’ordonnanceur de processus, et le dispatcher. Quand ces derniers sont-il exécutés ? Qu’est-ce-que la préemption et quels sont ses avantages et inconvénients ?}
{}
