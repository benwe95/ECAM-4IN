%--- 21 -------------------------------%
\item\quest{Définissez et expliquez le concept de défaut de page. Quand un défaut de page se produit-il ? En quoi altèrent-ils les performances du système ?}
{}


%--- 22 -------------------------------%
\item\quest{Définissez, expliquez et comparez les différentes stratégies que l’on peut mettre en place pour allouer les cadres.}
{}


%--- 23 -------------------------------%
\item\quest{Définissez la notion d’écroulement et expliquez en quoi elle rend un système instable. Comment
peut-on détecter et se protéger d’un écroulement ?}
{}


%--- 24 -------------------------------%
\item\quest{Définissez la notion de fichier et expliquez comment le système d’exploitation les gère. Citez et expliquez des exemples d’opérations qu’il est possible de réaliser sur des fichiers.}
{}


%--- 25 -------------------------------%
\item\quest{Expliquez comment un disque peut être structuré en partition/volume/systèmes de fichiers.
Citez, expliquez et comparez différentes manières de concevoir une structure en répertoires.}
{}


%--- 26 -------------------------------%
\item\quest{Définissez ce qu’est un système de fichiers et les informations que le système d’exploitation
retient pour l’organiser. En quoi consiste le montage d’un système de fichiers et qu’est-ce-que le
système de fichiers virtuel ?}
{}


%--- 27 -------------------------------%
\item\quest{Citez, expliquez et comparez les différentes méthodes d’allocation utilisées pour stocker les
fichiers sur le disque.}
{}


%--- 28 -------------------------------%
\item\quest{Citez et expliquez les deux types de formatage d’un disque qu’il est possible de réaliser. Quels sont les buts précis de chacun de ces deux formatages ?}
{}


%--- 29 -------------------------------%
\item\quest{Citez, expliquez et comparez les différentes structures RAID qu’il est possible de mettre en
oeuvre. Dans quel cas concret utiliseriez-vous l’une ou l’autre structure ?}
{}


%--- 30 -------------------------------%
\item\quest{Définissez ce qu’est un driver de périphérique. Où ce dernier intervient-il et avec quels autres composants interagit-il afin de permettre à un utilisateur d’exploiter un périphérique.}
{}
