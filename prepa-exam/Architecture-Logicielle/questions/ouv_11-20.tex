%--- 11 ----------------------------------------------------------
\item\qouv{Quelles sont les différentes étapes de l’appel d’un service web ?}
{}

%--- 12 ----------------------------------------------------------
\item\qouv{Expliquez brièvement les six contraintes d’une architecture REST ?}
{}

%--- 13 ----------------------------------------------------------
\item\qouv{Exposez brièvement les différences entre IaaS, PaaS et Saas.}
{}

%--- 14 ----------------------------------------------------------
\item\qouv{Exposez brièvement les différences entre IaaS, PaaS et Saas.}
{}

%--- 15 ----------------------------------------------------------
\item\qouv{Comment se calcule la complexité Fan-in Fan-out et quels sont ses avantages et inconvénients ?}
{La complexité "Fan-in Fan-out" est un \textit{métrique} qui se base sur le flux des données locales (in=params et out=valeurs retour).
\paragraph{}
Deux variantes existent pour calculer cette valeur de complexité:
\begin{enumerate}
\item \textbf{Henry et Kafura}: $HK = Length * (Fan_{in} * Fan_{out})^{2}$
\item \textbf{Shepperd}: $S = (Fan_{in} * Fan_{out})^{2}$
\end{enumerate}

\paragraph{UTILITE [...]}
}


%--- 16 ----------------------------------------------------------
\item\qouv{Définissez les notions de couplage afferent et efferent. Comment construit-on l’instabilité à partir
de ces métriques.}
{}

%--- 17 ----------------------------------------------------------
\item\qouv{Qu’est-ce-que la distance from main sequence et que permet-elle de mesurer ?}
{}

%--- 18 ----------------------------------------------------------
\item\qouv{Définissez brièvement les principes DRY et WET.}
{}

%--- 19 ----------------------------------------------------------
\item\qouv{Définissez le concept d’orthogonalité. En quoi améliore-t-il la qualité d’un logiciel ?}
{}

%--- 20 ----------------------------------------------------------
\item\qouv{Expliquez brièvement le principe du code traçant.}
{}