%--- 31 ----------------------------------------------------------------------------------
\item\vf{Selon la définition du NIST, l’une des caractéristiques du cloud est d’avoir une très grande
élasticité et une adaptation très rapide.}
{}
{}

%--- 32 ----------------------------------------------------------------------------------
\item\vf{Avec l’IaaS, le client doit gérer de lui-même l’installation et la mise à jour de son application.}
{}
{}

%--- 33 ----------------------------------------------------------------------------------
\item\vf{Google docs est un service dans le cloud de type PaaS.}
{}
{}

%--- 34 ----------------------------------------------------------------------------------
\item\vf{La complexité software de Halstead offre une mesure de la structure d’un code.}
{???}
{La complexité de Halstead se base sur l'implémentation \textbf{effective} d'un programme. Elle utilise le nombre d'opérateurs et d'opérandes uniques/totaux pour en déduire une série de propriétés: \textit{taille du vocabulaire}, \textit{longueur du programme}, \textit{volume d'information (V en bits)}, \textit{difficulté (D)}, \textit{efforts (E)}, \textit{temps d'implémentation (T)}, ...}
\paragraph{}
Ce modèle simpliste peut facilement être mis en place et permet principalement une \textbf{comparaison relatives} entre plusieurs programmes.
\paragraph{}
Cependant, il ne s'agit donc pas de prédictions mais bien de mesures obtenues sur base du programme développé. De plus, les valeurs sont fortement dépendantes du langage utilisé (ex: les opérateurs varient selon que l'on soit en bas niveau vs haut niveau, d'un langage à l'autre,...)

%--- 35 ----------------------------------------------------------------------------------
\item\vf{La complexité cyclomatique offre une mesure de la structure d’un code.}
{???}
{La complexité cyclomatique mesure le nombre d'\textbf{instructions de décision}, autrement dit le nombre de chemins possibles dans une fonction.
\paragraph{}
Ce métrique, facile à calculer et à mettre en oeuvre, permet d'évaluer la facilité de maintenance d'un code (au plus la complexité est grande, au plus le code devient difficile à comprendre) ainsi que d'identifier les zones pour lesquelles les efforts de tests seront les plus importants (là où la comp. est grande).
\paragraph{}
Cependant, ce modèle n'évalue pas la complexité des données.
}
