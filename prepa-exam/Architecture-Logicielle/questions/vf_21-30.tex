%--- 21 ----------------------------------------------------------------------------------
\item\vf{Un bon système distribué doit être plus fiable que le même système en centralisé unique.}
{\vrai}
{Un système distribué étant plus difficile à mettre en place, si celui-ci n'est pas plus fiable que son équivalent en centralisé unique alors il n'a que peu d'intérêt.


}

%--- 22 ----------------------------------------------------------------------------------
\item\vf{Dans une architecture client-serveur, on a toujours un unique serveur et un ou plusieurs clients.}
{}
{}

%--- 23 ----------------------------------------------------------------------------------
\item\vf{Une architecture de type broker peut être vue comme orientée service.}
{}
{}

%--- 24 ----------------------------------------------------------------------------------
\item\vf{Dans le cadre de services web, UDDI est un langage de description de services.}
{}
{}

%--- 25 ----------------------------------------------------------------------------------
\item\vf{Dans le cadre de services web, WSDL est un langage de description de services.}
{}
{}

%--- 26 ----------------------------------------------------------------------------------
\item\vf{Le standard REST définit les règles précises à appliquer pour obtenir des services web RESTful.}
{}
{}

%--- 27 ----------------------------------------------------------------------------------
\item\vf{L’architecture d’un compilateur est typiquement orientée flux de données.}
{}
{}

%--- 28 ----------------------------------------------------------------------------------
\item\vf{L’architecture centrée données consiste en un data store passif et des clients actifs.}
{}
{}

%--- 29 ----------------------------------------------------------------------------------
\item\vf{L’architecture blackboard consiste en un data store passif et des clients actifs.}
{}
{}

%--- 30 ----------------------------------------------------------------------------------
\item\vf{Le cloud computing consiste simplement à installer des logiciels sur des serveurs plutôt que des
desktop stations ou laptops afin de les rendre accessibles à tout le monde via internet.}
{}
{}