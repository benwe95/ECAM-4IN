%--- 1 -------------------------------%
\item\vf{Dans une entreprise, les données vivent souvent plus longtemps que les logiciels.}
{\vrai}
{blabla}


%--- 2 -------------------------------%
\item\vf{Dans une entreprise, les logiciels vivent souvent plus longtemps que les données.}
{\faux}
{coucou}


%--- 3 -------------------------------%
\item\vf{Le passage du relationnel au NoSQL se fait généralement au profit d’une diminution des garanties
relatives à la consistence des données.}
{}
{}


%--- 4 -------------------------------%
\item\vf{Le passage du relationnel au NoSQL se fait généralement au profit d’une diminution de la
quantité de données stockables.}
{}
{}


%--- 5 -------------------------------%
\item\vf{Le passage du relationnel au NoSQL se fait généralement au profit de l’abandon de la possibilité
de lire des données de manière concurrente.}
{}
{}


%--- 6 -------------------------------%
\item\vf{Le passage du relationnel au NoSQL se fait généralement au profit d’une diminution des garanties
relatives à la persistance des données.}
{}
{}


%--- 7 -------------------------------%
\item\vf{Tout comme pour le relationnel, l’organisation des données en NoSQL suit un modèle mathéma-
tique rigoureux.}
{}
{}


%--- 8 -------------------------------%
\item\vf{Le NoSQL est particulièrement adapté à des traitements de données de type OLTP}
{}
{}


%--- 9 -------------------------------%
\item\vf{Le NoSQL est particulièrement adapté à des traitements de données de type OLAP.}
{}
{}


%--- 10 -------------------------------%
\item\vf{Toutes les bases de données de type NoSQL satisfont les propriétés ACID.}
{}
{}