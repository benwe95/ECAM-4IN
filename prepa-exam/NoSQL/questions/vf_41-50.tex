%--- 41 -------------------------------%
\item\vf{L’avantage de l’utilisation de colonnes plutôt que de lignes est d’offrir un meilleur taux de
compression des données stockées.}
{}
{}


%--- 42 -------------------------------%
\item\vf{L’avantage de l’utilisation de colonnes plutôt que de lignes est d’offrir de meilleures performances lors de la lecture de tous les enregistrements d’une table.}
{}
{}


%--- 43 -------------------------------%
\item\vf{Une base HBase peut servir d’input/output de MapReduce (Hadoop)}
{}
{}


%--- 44 -------------------------------%
\item\vf{Une base HBase peut servir de fichiers avec GFS (Google File System)}
{}
{}


%--- 45 -------------------------------%
\item\vf{Une base de données orientée graphe stocke deux collections d’agrégats appelés nœuds et arêtes.}{\vrai}
{Le modèle orienté-graphe stocke deux types d'éléments:
\begin{itemize}
\item[$\cdot$]des \textcolor{ltred}{\textsc{noeuds}}: représentent des entités quelconques avec des propriétés
\item[$\cdot$]des \textcolor{ltred}{\textsc{arêtes}}: représentent des relations UNIDIRECTIONNELLES entre deux noeuds et possèdent une propriété. (\textit{Rem: pour faire un lien bidirictionnel, il faut donc nécessairement utiliser deux arêtes}
\end{itemize}

\paragraph{}
Puisqu'il n'y a pas de restriction sur les types des noeuds/relations, le graphe représente une collection hétérogène d'information dont l'avantage principal est de permettre une \textbf{traversée rapide} des relations, sans que celles-ci ne doivent être reconstruites.

\paragraph{}
Cette architecture est particulière utile dans les cas suivants:
\begin{itemize}
\item[$\cdot$]les collections \textit{très riche de liens} entre les entités
\item[$\cdot$]routage/dispatching et localisation
\item[$\cdot$]moteur de recommandations automatiques
\end{itemize}
}


%--- 46 -------------------------------%
\item\vf{La suppression d’un nœud dans une base de données orientée graphe implique la suppression de
toutes les relations partant et arrivant sur ce nœud.}
{\vrai}
{
Il ne peut \textbf{pas exister de lien mort} dans le graphe, autrement dit la suppression d'un noeud implique nécessairement de supprimer toutes les arêtes adjacentes.
}


%--- 47 -------------------------------%
\item\vf{Il est impossible de stocker une liste de personnes dans une base de données orientée graphe}
{\faux}
{
On distingue différents \textit{types de relations} dans le graphe dont une qui permet de représenter une \textbf{liste chaînée}. La relation est un "pointeur" vers l'élement suivant de la liste.
\begin{figure}[h!]
\center\includegraphics[scale=.3]{images/graphe-relation-liste}
\caption{Représentation d'une liste chaînée en modèle orienté-graphe \cite{ref1}}
\end{figure}
}


%--- 48 -------------------------------%
\item\vf{SPARQL est un langage de requêtes générique permettant d’interroger n’importe quelle base de
données NoSQL.}
{\faux}
{SPARQL (\textit{\textbf{S}PARQL \textbf{P}rotocol and \textbf{R}DF \textbf{Q}uery \textbf{L}anguage)} est un langage de requêtes adapté spécifiquement à la structure des graphes RDF.
\paragraph{}
Tout ceci fait partie de ce qui constitue plus communément le \textbf{Web Sémantique} qui est une extension standardisée du Web, pour l'utilisation de formats de données. Il s'agit donc du \textit{Web des données} qui permet de partager des informations structurées entre des applications. (\textit{« le Web sémantique fournit un modèle qui permet aux données d'être partagées et réutilisées entre plusieurs applications, entreprises et groupes d'utilisateurs »} W3C)
\paragraph{}
Dans ce contexte, le RDF (\textit{\textbf{R}esource \textbf{D}estcription \textbf{F}ormat}) est un \textbf{modèle de graphe} destiné à représenter de façon formelle les ressources web. Il est le langage de base du Web sémantique et permet de stocker les données sous forme de triplets "(Sujet) --- prédicat ---> (Objet)" pour marquer la relation entre deux entités.
\paragraph{}
Le SPARQL, quant à lui, est le langage de requêtes qui interroge les graphes RDF au moyen de quatre types d'instructions:
\begin{enumerate}
\item\textcolor{ltred}{\textsc{select}}: extraire un sous-graphe d'un graphe RD sous forme de table
\item\textcolor{ltred}{\textsc{construct}}: engendrer un nouveau graphe complétant un autre
\item\textcolor{ltred}{\textsc{ask}}: poser une question (True/False)
\item\textcolor{ltred}{\textsc{describe}}: extraire un graphe RDF
\end{enumerate}
}

%--- 49 -------------------------------%
\item\vf{Gremlin est un langage de requêtes générique permettant de décrire des traversées de graphe.}
{\vrai}
{Il est utilisé avec deux types de bases de données:
\begin{itemize}
\item[$\cdot$]OLTP: OrientDB,...
\item[$\cdot$]OLAP: Apache Giraph, Hadoop,...
\end{itemize}
\paragraph{}
Dans ce langage, une requête décrit la traversée à faire pour récupérer des informations, on effectue des opérations sur les noeuds.
}

%--- 50 -------------------------------%
\item\vf{Neo4j supporte les transactions ACID.}
{\vrai}
{Neo4j est un \textbf{système de gestion de graphes} qui supporte les transactions de type ACID et qui interroge les données au moyen du CQL (\textit{\textbf{C}ypher \textbf{Q}uery \textbf{L}anguage}), un langage à travers HTTP (requêtes de types: \textsc{create}, \textsc{match} et \textsc{set}).
\paragraph{}
Il permet d'associer aux noeuds et aux arêtes des \textit{propriétés} sous forme de paires "clé-valeur" et d'ajouter des \textit{labels} aux noeuds, ce qui permet de les regrouper en catégories.
\paragraph{}
Le résultat d'une requête est un \textit{chemin}, càd une séquence de noeuds avec des relations.
}