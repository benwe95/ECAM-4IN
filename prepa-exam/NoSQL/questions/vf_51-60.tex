%--- 51 -------------------------------%
\item\vf{Les bases de données orientée graphe sont très adaptée pour le sharding.}
{}
{}


%--- 52 -------------------------------%
\item\vf{OrientDB offre la possibilité d’utiliser le sharding de données.}
{}
{}


%--- 53 -------------------------------%
\item\vf{Une approche pessimiste de la consistence des données consiste à se limiter à un serveur unique
pour le stockage des données.}
{}
{}


%--- 54 -------------------------------%
\item\vf{Garantir la consistence de lecture empêchera tout conflit de type write-write.}
{}
{}


%--- 55 -------------------------------%
\item\vf{Garantir la consistence de mise à jour empêchera tout conflit de type read-write.}
{}
{}


%--- 56 -------------------------------%
\item\vf{Garantir la consistence de réplication est impossible avec un système peer-to-peer.}
{}
{}


%--- 57 -------------------------------%
\item\vf{Si mes données sont répliquées sur quatre nœuds, avec W = 2, il suffit de lire deux nœuds pour
lire l’information la plus à jour.}
{}
{}


%--- 58 -------------------------------%
\item\vf{Si mes données sont répliquées sur quatre nœuds, il suffit d’impliquer W = 2 nœuds dans
l’écriture pour assurer une consistence des données.}
{}
{}


%--- 59 -------------------------------%
\item\vf{L’utilisation d’un timestamp comme version stamp est moins lourd à déployer que d’utiliser un
GUID (Globally Unique Identifier).}
{}
{}


%--- 60 -------------------------------%
\item\vf{L’utilisation d’un GUID (Globally Unique Identifier) comme version stamp est moins lourd à
déployer que d’utiliser un timestamp.}
{}
{}

