%--- 21 -------------------------------%
\item\vf{Dans une base de données clé-valeur, il est généralement prévu de rechercher toutes les clés dont
les valeurs satisfont une certaine propriété.}
{}
{}


%--- 22 -------------------------------%
\item\vf{Il est possible d’imposer des contraintes sur les domaines des valeurs des paires clé-valeur d’une base de données clé-valeur.}
{}
{}


%--- 23 -------------------------------%
\item\vf{Distribuer les données sur un cluster de machines fait partie des éléments mis en place dans le
monde NoSQL.}
{}
{}


%--- 24 -------------------------------%
\item\vf{Il est possible de faire du sharding de données pour une base de données se trouvant sur un
machine unique.}
{}
{}


%--- 25 -------------------------------%
\item\vf{Le sharding permet de récupérer les données en cas de corruption grâce à un stockage redondant
de ces dernières sur plusieurs serveurs pouvant être physiquement à des endroits différents.}
{}
{}


%--- 26 -------------------------------%
\item\vf{Réplication de données et sharding sont incompatibles.}
{}
{}


%--- 27 -------------------------------%
\item\vf{La réplication master-slave offre la propriété de résilience à la lecture.}
{}
{}


%--- 28 -------------------------------%
\item\vf{En utilisant une réplication master-slave, les données deviennent complètement inaccessibles une
fois que le master tombe.}
{}
{}


%--- 29 -------------------------------%
\item\vf{La consistence des données est plus compliquées à garantir avec une réplication master-slave
qu’avec une réplication peer-to-peer.}
{}
{}


%--- 30 -------------------------------%
\item\vf{La consistence des données est plus compliquées à garantir avec une réplication peer-to-peer
qu’avec une réplication master-slave.}
{}
{}
