%--- 61 -------------------------------%
\item\vf{L’utilisation d’un GUID (Globally Unique Identifier) comme version stamp permet de retrouver
la version la plus récente d’une donnée.}
{\faux}
{Le GUID ne permet pas de trouver la version la plus récente d'une donnée par comparaison, il permet simplement de [...] \small{\textit{identifier la version de changement???}}
}


%--- 62 -------------------------------%
\item\vf{Le write optimiste est très cher à implémenter dans un modèle clé-valeur.}
{}
{}


%--- 63 -------------------------------%
\item\vf{Dans une base de données orientée colonnes, les données transitent par plusieurs espaces de
stockage avant leur destination finale permanent.}
{}
{}


%--- 64 -------------------------------%
\item\vf{Il est possible d’utiliser de la réplication master-slave avec une base de données orientée graphe pour rendre les lectures plus performantes.}
{}
{}


%--- 65 -------------------------------%
\item\vf{Les bases de données orientée documents permettent d’effectuer des transactions atomiques au
niveau d’un document.}
{}
{}


%--- 66 -------------------------------%
\item\vf{Les bases de données orientée documents permettent d’effectuer des transactions atomiques au
niveau d’une collection.}
{}
{}


%--- 67 -------------------------------%
\item\vf{On peut changer le modèle d’une base de données NoSQL entre clés-valeurs, colonnes et
documents, tout en garantissant exactement le même ensemble de propriétés.}
{}
{}


%--- 68 -------------------------------%
\item\vf{Le passage vers le NoSQL permet de se passer des ORMs.}
{}
{}


%--- 69 -------------------------------%
\item\vf{Le NoSQL est très adapté pour stocker des données très uniformes.}
{}
{}

%--- 70 -------------------------------%
\item\vf{Le passage du relationnel au NoSQL rend les calculs à effectuer sur les données plus lents suite à un éventuel cout de transfert des données au sein du cluster.}
{}
{}
