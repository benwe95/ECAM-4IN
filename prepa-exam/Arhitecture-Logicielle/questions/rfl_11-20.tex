%--- 11 -----------------------------------------------------------------------
\item\qouv{Afin d’évaluer la complexité d’un système logiciel, on peut procéder à des mesures sur le
code de ce dernier. En particulier, on peut utiliser les métriques de Halstead, McCabe et Henry
ou Kafura/Shepperd pour mesurer différents types de complexité. Quels sont les aspects de complexité mesurés par ces métriques ? Comment sont elles calculées ? Discutez de la pertinence
des mesures ainsi réalisées, par rapport aux variables prises en compte et de l’utilité que l’on
peut faire de ces métriques.}
{}

%--- 12 -----------------------------------------------------------------------
\item\qouv{En quoi l’architecture en microservices permet-elle de suivre le principe de responsabilité
unique (SRP). Expliquez et argumentez en mentionnant les bénéfices d’une telle architecture.
En particulier, illustrez votre réponse en utilisant la notion d’architecture serverless et à l’aide
d’exemples concrets.}
{}
