%--- 11 ----------------------------------------------------------------------------------
\item\vf{Le choix d’architecture peut avoir une influence sur la sécurité du système logiciel.}
{\vrai}
{L'architecture logicielle influence fortement les \textbf{propriétés} du système dont sa sécurité. Par exemple, il est plus difficle d'assurer la sécurité d'un système s'il est distribué (modèle en niveaux) que s'il est sur une unique machine.
\paragraph{}
Comme un choix d'architecture est quasi définitif, il y a des points de \textbf{non retour} lors du développement du système, la phase d'analyse du projet est donc primordiale.
\paragraph{Remarque:} il est \textit{impossible} d'optimiser toutes les propriétés d'un système (performance, sécurité, disponibilité, maintenabilité, fiabilité, tolérance aux pannes, comptabilité,...) car l'amélioration d'un critère se fait bien souvent au détrimant d'un autre. Des choix doivent donc être faits ( = COMPROMIS) quant aux critères les plus pertinents suivant l'application du système.

\paragraph{}
Afin d'améliorer la qualité d'un logiciel, une solution consiste à créer plus de tests unitaires pour couvrir le code au plus possible et effectuer soi-même d'avantage de "\textit{tests à la main}" pour évaluer tous les scénarios possibles d'utilisation. 
\paragraph{}Il est à noter que la \textit{qualité logicielle} est une notion très vague. Il faut se baser sur des critères (ex, norme ISO/CEI 9126: capacité fonctionnelle, fiabilité, facilités d'utilisation, performance, maintenabilité) et définir la façon de les mesurer pour avoir une évaluation qui a du sens.
}

%--- 12 ----------------------------------------------------------------------------------
\item\vf{Un design pattern est un template de code applicable automatiquement étant donné la spécifica-
tion d’une procédure/fonction/méthode.}
{}
{}

%--- 13 ----------------------------------------------------------------------------------
\item\vf{Le design pattern du GoF Singleton permet de créer des instances d’une classe une à la fois.}
{}
{}

%--- 14 ----------------------------------------------------------------------------------
\item\vf{Le design pattern du GoF Builder est de type construction.}
{}
{}

%--- 15 ----------------------------------------------------------------------------------
\item\vf{Pour appliquer le design pattern du GoF Facade, il faut impérativement rendre toutes les
procédures/fonctions/méthodes des sous-systèmes à cacher privées.}
{}
{}

%--- 16 ----------------------------------------------------------------------------------
\item\vf{Le design pattern du GoF Template permet d’implémenter un algorithme incomplets avec des
« trous » à remplir (hooks).}
{}
{}

%--- 17 ----------------------------------------------------------------------------------
\item\vf{La programmation impérative met l’accent sur le comment un programme fonctionne.}
{}
{}

%--- 18 ----------------------------------------------------------------------------------
\item\vf{La programmation déclarative met l’accent sur le comment un programme fonctionne.}
{}
{}

%--- 19 ----------------------------------------------------------------------------------
\item\vf{La programmation fonctionnelle est plus proche de l’impérative que de la déclarative.}
{}
{}

%--- 20 ----------------------------------------------------------------------------------
\item\vf{La programmation fonctionnelle est plus éloignée de l’impérative que de la déclarative.}
{}
{}