%--- 1 -----------------------------------------------------------------------
\item\qouv{Il existe un lien fort entre l’architecture et la qualité d’un système logiciel. En particulier,
améliorer la qualité logicielle peut se faire en choisissant une architecture pertinente par rapport
au cahier des charges du système à développer. Expliquez en donnant des exemples concrets, et
argumentez.}
{}

%--- 2 -----------------------------------------------------------------------
\item\qouv{L’architecte logiciel est le garant de l’intégrité conceptuelle du système logiciel. De quoi
s’agit-il ? Quelles sont les différents éléments auquel il devra faire attention tout au long de la
durée de vie du logiciel ? Illustrez vos explications avec des exemples concrets, et argumentez.}
{}

%--- 3 -----------------------------------------------------------------------
\item\qouv{Décrivez brièvement les six principaux styles d’architecture suivant en donnant, pour chacun,
les avantages et inconvénients et un exemple concret utilisant ce style. Comparez ensuite ces
styles et déterminez une procédure qui permettrait à un architecte de se diriger vers le style le
plus adéquat étant donné un système logiciel à concevoir.
Centrée sur les données, flot de données, en couches, en niveaux, invocation implicite et MVC}
{}

%--- 4 -----------------------------------------------------------------------
\item\qouv{Le pattern d’implémentation argue qu’il faut viser l’excellence en programmation en suivant
les trois valeurs importantes que sont la communication, la simplicité et la flexibilité. En vous
appuyant sur des exemples de systèmes logiciel avec un choix d’architecture le plus adapté,
discutez à partir des avantages de l’architecture choisie de pourquoi elle permet de tendre vers
l’excellence.}
{}

%--- 5 -----------------------------------------------------------------------
\item\qouv{Un mauvais système logiciel peut se détériorer avec le temps avec pour conséquence qu’il
deviendra cher et difficile, voir impossible, à maintenir. L’une des causes majeures est que le
code a été sous-ingénierié. Qu’est-ce-que cela signifie-t-il et quelles sont les principales raisons
pouvant mener à un tel code ? Quelles bonnes pratiques, tant au niveau du code qu’au niveau de
l’architecture, peuvent aider à éviter un code sous-ingénieré ? Argumentez.}
{}

%--- 6 -----------------------------------------------------------------------
\item\qouv{Lorsqu’il s’agit de choisir un ou des langage(s) de programmation concret pour développer un
système logiciel, quelles sont les questions à se poser ? Comment peut-on organiser et structurer
la réflexion qui va guider vers le choix de langage ? Expliquez et argumentez.}
{}

%--- 7 -----------------------------------------------------------------------
\item\qouv{Dans les architectures orientée-interaction le système logiciel est découpé en trois partitions
principales : données, contrôle et vue. Comparez les trois grandes familles de modèles MV*
(MVC, MVVM et MVP) en identifiant comment les trois partitions sont organisées et identifiez
les avantages et inconvénients des différentes architectures existantes des trois familles.}
{}

%--- 8 -----------------------------------------------------------------------
\item\qouv{Les architectures de type broker et orientée-service possèdent une série de points communs,
notamment qu’elles permettent toutes deux de partager des services. Mais en quoi diffèrent-elles ?
Quels sont les avantages et inconvénients de ces deux types d’architecture ? Pour quel type
d’applications l’une ou l’autre sera-t-elle plus adaptée ? Argumentez.}
{}

%--- 9 -----------------------------------------------------------------------
\item\qouv{Les architectures prédominantes ont évolué avec les changements business partant de solutions
monolithiques vers des solutions actuellement orientées services. Expliquez comment cette
évolution s’est passée en reprenant les points forts et faibles de chacune des architectures de la
ligne du temps suivante et argumentez.}
{}

%--- 10 -----------------------------------------------------------------------
\item\qouv{Il y a trois principales architectures orientées données que sont le batch séquentiel, les pipes
et filtres et le contrôle de processus. Quels sont les points communs et différences entre ces trois
architectures, les avantages et inconvénients. Illustrez votre réponse à partir d’exemples concrets.
Quelles sont les questions que l’on pourrait se poser afin d’orienter son choix vers l’une des
trois architectures sachant qu’on a à réaliser un système logiciel qui doit traiter des données ?
Argumentez.}
{}